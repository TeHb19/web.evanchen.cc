\documentclass[11pt]{scrartcl}
\usepackage[sexy,von]{evan}

\renewcommand{\vonenvname}{inlineprob}
\ohead{\bfseries\footnotesize Thoughts on Problem Writing}

\newtheoremstyle{inlined}% name of the style to be used
  {8pt}% measure of space to leave above the theorem. E.g.: 3pt
  {3pt}% measure of space to leave below the theorem. E.g.: 3pt
  {\addtolength{\leftskip}{5ex}}% name of font to use in the body of the theorem
  {0pt}% measure of space to indent
  {\bfseries}% name of head font
  {.}% punctuation between head and body
  { }% space after theorem head; " " = normal interword space
  {}% Manually specify head

\theoremstyle{inlined}
\newtheorem{inlineprob}[theorem]{Example}

\begin{document}
\title{From the Author's Side: Thoughts on Problem Writing}
\subtitle{maybe with some advice thrown in}
\date{14 July 2021}
\maketitle

\epigraph{It used to be said that a Broadway musical was
  a success if the audience left the theater whistling the tunes.
  I want to see contestants leave the Putnam whistling the problems.
  They should be vivid and striking enough to be shared with roommates and teachers.}
  {Putnam problems committee greeting letter}

This is a piece about the problem development aspects of math contests.
Some of the sections will be philosophical,
with ideas on things like what makes a ``good'' problem.
Others will be practical, focusing on fine details of writing.

\tableofcontents
\pagebreak

\section{What makes a good problem?}
In my opinion, this is both an Art and a Science.
The Art is making the problem aesthetically attractive.
The Science is making the problem a good separator.
(There is a bit of overlap and correlation here.)

There are examples of criteria in each category
you can use to judge a problem,
and I think a lot of the variance in ``taste'' is due to different weights.

\subsection{Examples of Art}
I think in many cases, you can think of the aesthetics
of the problem in terms of the statement and the solution.
The elegance may be correlated, but is not the same.

\begin{itemize}
\ii \alert{Clean or natural statements}:
You might like the problem statement to be clean and easy to understand,
something that a contestant would naturally want to work on.
Walls of text or dense notation are generally not desirable;
one-sentence problems are often inviting.
\von{USAMO 2004/3}

\ii \alert{Fundamental statement}:
I'm not sure how to describe this,
but there's a sense in which a problem might be ``fundamental''
(more than just the problem statement being natural).
Perhaps it is a generalization of some classical facts.
\von{IMO 2017/6}
\von{USA TST 2007/3}

\ii \alert{Slick or short solution}:
Even though the problem appears (or is) challenging,
the solution turns out to be short.
There is one big ``a-ha'' moment.
Conversely, problems might score low on this scale
if the solution involves lots of computation or casework.
\von{ELMO 2017/3}

\ii \alert{Surprising}:
It's often desirable that the problem statement or solution
is \emph{counter-intuitive}, in the sense that it is
\emph{surprising} that the problem is true/provable,
or the answer runs contrary to one's expectations.
For example:
\von{January TST 2013/4}

\ii \alert{Inviting}:
Some problems almost seem to invite contestants to begin playing:
seeing the problem makes you naturally want to start
writing down some examples.
\von{TSTST 2016/5}

\ii \alert{Low-tech solution}:
It is often desirable that the solution to a problem
is as elementary as possible (for example a geometry problem
using only angle chasing).
One famous example is the windmill problem,
where the solution looks almost infantile in simplicity
(but is quite hard to find).
\von{IMO 2011/2}

\ii \alert{Depth}: The problem might have generalizations,
or be connected to other fields in some way.
Sometimes the solution might be extended to eliminate
entire classes of problems.
This is part of the reason people might find geometry problems appealing:
the diagram is often ``rich'' in the sense that there are many
more things to be proven in a picture even once the original problem is solved.
\von{Shortlist 2016 G2}

\ii \alert{Multiple solutions}:
I think this a symptom of some criteria above
(namely, problems which are ``fundamental'' or ``deep'' often do),
but some people view having multiple solutions as nice in its own right.
Geometry problems are especially prone to
having multiple (synthetic) approaches.

\ii \alert{Sheer difficulty}:
Some problems are so hard that they become legendary.
\von{Shortlist 2003 A6}
\end{itemize}

\subsection{Examples of Science}
I should say a bit about the dichotomy between the Art and Science.
The Putnam greeting letter writes is that
``any problem solved by only one or two contestants is a failure,
no matter how beautiful it might be''.
I am sympathetic to this view (though many are not).

In my opinion, contests are not only for fun or beauty;
they are supposed to be an education experience for their participants.
A problem solved by no contestants (which has happened plenty of times)
may not be desirable for this reason --- it does no good
to have a beautiful problem that no one has time to attempt.
Conversely, a problem may be nice but simply too easy to use for a particular contest
(and thus maybe should be submitted to a different contest,
where it will be better appreciated).

Alas, the issue is not black and white: there is a balance
one has to strike between selecting problems for
aesthetic as well as their suitability for an exam.

\begin{itemize}
  \ii \alert{Appropriate difficulty}:
  As described above, problems should neither be gimme's
  nor be completely unapproachable.
  I want to point out that this may not even depend on the individual problem,
  but on the rest of the contest as a whole.
  For example, it was my opinion that
  IMO 2017/2 (functional equation) spoiled IMO 2017/3 (hunter and rabbit);
  both problems were popular in the jury's beauty contest,
  but it was not reasonable for a single 4.5-hour exam
  to have both problems, regardless of how easy the first problem was.
  \ii \alert{A good test}:
  There is a sense in which a problem is a \emph{good test}
  of whether you understand something well.
  Such problems are desirable since they make the scores
  on the contest more meaningful.
  For example, the following problem from PUMaC is one of my
  favorites since it is an excellent test of understanding
  the Chinese Remainder Theorem.
  \von{PUMaC 2013 N7}
  \ii \alert{Instructive}:
  Similar to the previous point,
  there are problems that you \emph{learn} a lot from working on.
  For example, if I am trying to teach a student about
  switching the order of summation,
  I basically always begin with the following problem.
  \von*{Phi Floor}
  \ii \alert{Possible to approach}:
  It is nice if contestants can still make some progress
  on a difficult problem even if they cannot solve it completely.
  This is a much deeper engagement with the problem
  than simply sitting there drawing a blank.
  \von{IMO 2017/2}
  \ii \alert{Novel}: this is really hard.
  Vieta jumping was a really beautiful trick the first time it appeared,
  and I think this method is one that every student should see once in their life.
  But it garnered the universal contempt of the olympiad community
  after appearing repeatedly on contests until 2010 or so,
  to the point that contestants simply regurgitated the trick from memory.
  The first serving of a cake may be delicious; the tenth is probably not.
  I also like the following note from Bruce Reznick:
  \begin{quote}
     This leads to a tradeoff between familiarity and quality.
     We occasionally receive a complaint that a problem
     is not new \dots This is unfortunate, but probably inevitable.
     It would be easy to write an exam with twelve highly convoluted,
     certifiably original, and thoroughly uninteresting problems.
  \end{quote}
\end{itemize}

\section{Okay, but how do you write a problem?}
\renewcommand{\vonenvname}{example}
I don't have a good answer to this but can offer some examples
to give you a sense of my workflow.
In my case the process flows into two distinct parts:
\begin{itemize}
  \ii Constructing the ``core'' of the problem, and
  \ii Cosmetic changes to the problem.
\end{itemize}
The first step is the difficult part, and is quite discrete:
there is a big difference between having nothing
and having something that ``could'' appear on a contest
(and not too much of a grey area in between).
The next section will talk about how to do edits from a core.

\subsection{Backwards versus forwards}
I think roughly there's two ways to come up with the core of the problem:
\begin{itemize}
  \ii \vocab{Backwards}:
  start with the main idea of the solution,
  and try to work it into a reasonable problem statement.
  \ii \vocab{Forwards}: Start somewhere natural or interesting, and explore.
  If you have problem statement, try to solve it,
  but allow yourself to change the problem statement while doing this,
  in the hopes of getting a workable problem.
\end{itemize}

These two methods have almost opposite benefits and drawbacks.
Backwards problems have slick solutions but less interesting statements;
forwards problems have nice statements but less interesting solutions.
So the challenge is to try to get both a nice solution and a nice statement.
Fortunately it is possible to work both forwards and backwards
while trying to invent a problem, and so the issue is not black-and-white.

\subsection{Two examples of (mostly) backwards problems}
An example of a problem I wrote backwards was ELMO 2013/2.
\von{ELMO 2013/2}
I came up with this problem because I wanted an inequality
which was solved by applying AM-GM,
but with the unknowns as the weights:
\[ \sum_{\text{cyc}} \frac{a}{a+b+c} \cdot x \ge \prod x^{\frac{a}{a+b+c}}. \]
I soon realized $x = a^r$ would yield something interesting,
and then that I could hide everything by
requiring $a^{r-1} + b^{r-1} + c^{r-1} = a+b+c$.
Then I chose $r = -6/7$, to get seventh roots.

Another problem I wrote backwards was JMO 2016/2.
\von{JMO 2016/2}
I invented this problem one day while lying
on the bed in my dorm room: I had been mindlessly
thinking about powers of $5$ and came to notice that
\[ 5^8 \equiv 5^4 \equiv 625 \pmod{10^4} \]
which meant that $5^8$ should end in $0625$.
It then occurred to me that using this process I would
be able to generate as many $0$'s as I wanted near the end of $5^n$.
And that's exactly how I made the problem,
but with some aesthetic choices (such as $n < 10^6$
as $6$ consecutive zeros, which somehow ``lines up'').

\subsection{Two examples of (mostly) forwards problems}
Nearly every geometry problem I have created is written forwards.
Here are two illustrative examples done in full:
\begin{center}
   \url{http://web.evanchen.cc/handouts/Writing-Oly-Geo/Writing-Oly-Geo.pdf}
\end{center}


\section{Editing the problems}
Once you have the core of a problem,
it's often worthwhile to make other design decisions
on how to present the idea.

\subsection{Real math edits}
These are edits that change the math content of the problem.

As an example, here is again the JMO 2016/2 mentioned earlier.
\von{JMO 2016/2}
The ``core'' of the problem (i.e.\ the hard part to come up with)
was the idea of generating leading zeros
by showing $5^n$ is congruent to smaller power of $5$ modulo $10^n$.
But after this, I still had to decide on some parameters, such as
\begin{itemize}
   \ii whether to phrase the question as yes/no or prove existence,
   \ii specifically asking for six leading zeros
   (rather than $2016$ or a variable number leading zeros).
   \ii the choice of requiring $n < 10^6$
   (rather than allowing any $n$).
\end{itemize}

The changes can be more substantial than that.
As an example, here is the diagram for USA TST 2017/5,
proposed by Danielle Wang and me.
\von{USA TST 2017/5}
\begin{center}
\begin{asy}
size(12cm);
pair A = dir(110);
pair B = dir(210);
pair C = dir(330);
pair I = incenter(A, B, C);
pair M = dir(270);
pair J = 2*M-I;
pair D = foot(J, B, C);
pair E = foot(A, B, C);

pair F = IP(CP(J, D), unitcircle);
pair G = OP(CP(J, D), unitcircle);

filldraw(unitcircle, opacity(0.1)+lightblue, lightblue);
draw(arc(J,abs(D-J),-30,210), lightblue);

pair L = midpoint(A--E);
pair V = extension(G, D, L, L+B-C);
pair N = extension(F, D, L, L+B-C);
draw(A--B--C--cycle, lightblue);
draw(A--E, lightblue);
draw(G--V, lightblue);
draw(N--F, lightblue);
filldraw(E--V--A--N--cycle, opacity(0.1)+lightgreen, heavygreen);
draw(V--N, heavygreen);
pair H = extension(G, F, B, C);
draw(C--H--G, lightblue);

filldraw(circumcircle(B, I, C), opacity(0.1)+lightred, lightred);
pair Y = IP(circumcircle(B, I, C), circumcircle(D, F, G));
pair Z = OP(circumcircle(B, I, C), circumcircle(D, F, G));
pair T = -D+2*foot(J, D, I);
draw(T--H, blue);
draw(Z--H, blue);
draw(L--T, blue);
draw(Y--I--Z, blue);

dot("$A$", A, dir(A));
dot("$B$", B, dir(B));
dot("$C$", C, dir(C));
dot("$I$", I, dir(I));
dot("$M$", M, dir(M));
dot("$J$", J, dir(J));
dot("$D$", D, dir(D));
dot("$E$", E, dir(E));
dot("$F$", F, dir(270));
dot("$G$", G, dir(270));
dot("$L$", L, dir(L));
dot("$V$", V, dir(V));
dot("$N$", N, dir(N));
dot("$H$", H, dir(H));
dot("$Y$", Y, dir(Y));
dot("$Z$", Z, dir(Z));
dot("$T$", T, dir(T));
\end{asy}
\end{center}
This problem is interesting as a case study because
the core of the problem came from Danielle,
but the editing of the problem was all done by me.
Danielle's discovery was that the point $I$
was on the $D$-symmedian of $\triangle DFG$.
This was nice, but defining the $D$-symmedian would give away
too much of the problem, and was perhaps not so elegant.

So, I suggested the idea that line $DI$ also passed through
the midpoint of the $A$-altitude;
taking a projection from $D$ gives the rhombus\footnote{For
the record, Danielle insisted that I name the rhombus ``$EVAN$''.
Not my idea.} formulation,
which (neatly) also eliminates the mention of $I$ from the problem.

\renewcommand{\vonenvname}{inlineprob}
Here's a couple suggestions for substantial math edits:
\begin{itemize}
  \ii \alert{Cut}.
  In general, given two formulations of a problem, I prefer to pick the shorter one.
  For example the rhombus in the earlier TST 2017 problem
  is much shorter than introducing the incenter $I$
  and then having to work with symmedians.

  \ii \alert{Specialize}.
  If a problem holds for all $n$ but each individual $n$
  feels about the same, consider replacing $n$ with $2017$,
  so that one does not need to write ``let $n$ be a positive integer''.
  (Some people give the opposite advice, so it may be a matter of taste.)
  For example, in the JMO problem, I could have asked to
  show $5^n$ has $k$ consecutive zeros for some $n < f(k)$ less than something,
  but the problem seems more attractive if I picked $k = 6$, $f(k) = 10^6$.
  Another example is the following USAMTS problem of mine,
  which holds for many choices of constants in place of $5/2$
  but becomes more cumbersome to state and solve then:
  \von{USAMTS 5/1/28}
  This has the added benefit that often you can remove
  awkward quantifier issues (``for each $n$, find the minimum\dots'')
  and also eliminate questions about ``simplest form''
  (for problems in which the answer is a function of $n$);
  these type of issues are addressed in a later step.

  \ii \alert{Clean up ``answer extractions''}.
  Often the ``natural'' thing to ask for is too awkward to really state.
  For example, consider the following problem of mine.
  \von[Fall 2013]{OMO 2013 F22}
  The ``original'' problem really reads more like:
  \begin{quote}
     Find all $M$ for which $n^n \equiv 1 \pmod M \iff n \equiv 1 \pmod M$.
  \end{quote}
  but the answer does not fit a short-answer format:
  the answer is that whenever a prime $q$ divides $M$, so should $q-1$.
  So instead, I asked for which $m$ the number $M = 2013m$ worked.
  By then embedding this directly into the problem (without introducing $M$ itself),
  it become possible to get a one-sentence problem.
\end{itemize}

You should generally not be surprised if the core changes
a bit during this process. Writing is rewriting.

\subsection{Push the core as far as it will go}
In particular, when you have the core of an idea for a problem,
one thing that's useful to do is to try and squeeze as much as you can out of it.
Look at the present problem and method of proof.
Are there more general situations in which the proof could work?
Or, are there more specific situations in which stronger hypotheses might be true?
The final version of the problem doesn't necessarily have to be the hardest possible,
but you should at least do enough due diligence
so that you are \emph{aware} of the different possible formulations of the problem,
and can pick the one you like most.

One example of a \emph{failure} to do this is the following problem from TSTST 2020:
\von{TSTST 2020/3}
In fact, $n$ is not needed, and the following generalization\footnote{Actually,
   this generalization was discovered by Luke Robitaille, \emph{while taking the contest}.}
could have just as easily been asked:
\begin{quote}
   We say a nondegenerate triangle whose angles have
   measures $\theta_1$, $\theta_2$, $\theta_3$ is \emph{quirky}
   if there exists integers $r_1$, $r_2$, $r_3$, not all zero,
   such that \[ r_1 \theta_1 + r_2 \theta_2 + r_3 \theta_3 = 0. \]
   Find all quirky triangles whose sides are integers
   in arithmetic progression.
\end{quote}

I don't know if I would have used this version had I known about it:
it is potentially too difficult.
But that's beside the point --- I wanted to at least
\emph{know} that this was in the possibility space,
while crafting the proposal in its final form.

Of course, you probably will still miss things from time to time. It happens.
Having co-authors or test-solvers can help with this.

\subsection{Copy editing}
After substantial edits are done,
one should do copy edits to ensure that the problem statement
finally given is clear and complete.

Here are the most common edits I have to make:
\begin{itemize}
   \ii Do not start sentences with math mode.
   \begin{itemize}
      \ii Bad: ``$I$ is the incenter of triangle $ABC$.''
      \ii Good: ``Let $I$ be the incenter of triangle $ABC$.''
   \end{itemize}

   \ii Use punctuation properly at the end of displayed equations.
   Display equations which are too large to fit inline.
   (Using \verb+\dfrac+ is almost always a mistake;
   if it's not readable inline, it should be displayed.)

   \ii Define each variable before using it.
   Clarify the type of variables (integer, real, complex\dots).
   \begin{itemize}
      \ii Bad: ``If $2x+4 = 10$, find $3x+6$.''
      \ii Good: ``Let $x$ be a real number satisfying $2x+4=10$. Find $3x+6$.''
      \ii Bad: ``Find all functions $f$ satisfying $f(x+y)=f(x)+f(y)$.''
      \ii Good ``Find all functions $f \colon \RR \to \RR$
      such that for all real numbers $x$ and $y$,
      we have $f(x+y)=f(x)+f(y)$.''
   \end{itemize}

   \ii \textbf{Quantifiers}: this is tricky to get right
   and maybe half the copy edits I make are of this form.
   If you have a problem that has ``varying'' bits,
   for example a minimum or maximum, then you need to make it clear
   which things are allowed to vary and which things are fixed.

   A common mistake is to use the word ``let'',
   which is understood to fix variables in place.
   For example:
   \begin{itemize}
      \ii Bad: ``Let $a$, $b$, $c$ be positive real numbers with $a+b+c=3$.
      Find the minimum possible value of $a^2+b^2+c^2$.''
      \ii Okay: ``Given that $a$, $b$, $c$ are positive real numbers with $a+b+c=3$,
      find the minimum possible value of $a^2+b^2+c^2$.''
      \ii Good: ``Find the minimum possible value of $a^2+b^2+c^2$
      over all positive real numbers $a$, $b$, $c$ satisfying $a+b+c=3$.''
   \end{itemize}
   Sometimes if your problem is complex, this statement becomes verbose.
   \begin{quote}
      \dots Considering all possible starting positions,
      determine the latest possible moment at which the last ant falls off the checkerboard.
   \end{quote}

   \ii Edge cases:
   for example, if you write $\max S$ make sure that $S$ cannot be empty.
   This is particularly common in geometry problems,
   in which certain points may coincide or lines become parallel.
   For this reason I often specify that my triangles are scalene,
   because of some degeneracy if $AB=AC$.

   \ii For ``are'', ``is'', or similar constructs,
   be aware of cause confusion about whether something is ``exact'' or not.
   For example, ``Prove that there are two ways to do $X$''
   might be interpreted as ``exist at least two''
   or ``are exactly two''.
\end{itemize}

%\subsection{An extreme example of editing (before and after)}
%Here is an extreme example of a problem
%that had to be edited extensively to make the
%problem statement presentable for a competition.
%This is a bit of an unfortunate example because the problem
%statement became substantially less elegant,
%but it was necessary to ensure that a complete, correct problem
%statement was given.
%
%\begin{quote}
%   \textbf{Before}:
%   Given $n$ reduced positive fractions $\frac{a_i}{b_i}$,
%   find an $O(n)$ algorithm to compute the largest possible
%   fraction of the form \[ \frac{a_i+a_j}{b_i+b_j}. \]
%\end{quote}
%
%It seems like it would be impossible to pose
%this in a completely correct form on a math contest;
%the $O(n)$ is hard enough to define,
%but the real challenge is that it is hard to define
%what a ``step'' in an algorithm is without referencing a Turing machine.
%
%Nevertheless, after a lot of work we managed to find a formulation
%that we could propose for contest.
%Here is the final form.
%
%\von{HMIC 2017/1}

\section{Keep in touch}
Maybe the most useful advice I have (but the hardest to follow) is:
\begin{quote}
   Keep in touch with math contests!
\end{quote}
Keep looking at the IMO problems when they come out each year, etc.

Of all the contests I've had the privilege of working closely with,
it turns out the contest that receives the most problems
is not the USA TST or the USAMO, but rather the
\href{http://web.evanchen.cc/elmo/general.html}{ELMO}:
the sort of joke-pseudo-contest run by the students as annual tradition.
The problem quality\footnote{The organization of the ELMO is a different story!}
of ELMO shortlists is in my opinion
close to that of the IMO shortlist, except the ELMO is run by
a dozen bored students rather than by 100+ countries.

I don't think this is a coincidence.
%The strongest contestants each year treat math contests like
%a full-time job (with ``high-school student'' being almost a charade).
No one spends more time thinking about contest problems
than the actual contestants.
So it shouldn't be surprising they're also the ones with the
ripest problem ideas.

%In the words of Paul Graham,
%``it's hard to do a really good job on anything
%you don't think about in the shower''.

\end{document}
