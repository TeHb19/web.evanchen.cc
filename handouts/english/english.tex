\documentclass[11pt]{scrartcl}
\usepackage[sexy]{evan}

\begin{document}
\title{Remarks on English}
\subtitle{a.k.a.\ Advice for writing proofs}
\date{13 December 2023}
\maketitle

\epigraph{Exposition, criticism, appreciation, is work for second-rate minds.}
{G.\ H.\ Hardy}

\tableofcontents

\section{Examples of good and bad writing that we'll refer back to}
Following the examples-first philosophy, here are two problems with commentary
that we'll refer back to when we provide concrete advice.

\subsection{Introductory example: USAMO 2014 Problem 1}
This example is meant to be short and more accessible.
\begin{example}
  [USAMO 2014/1]
  Let $a$, $b$, $c$, $d$ be real numbers such that $b-d \ge 5$ and all zeros
  $x_1$, $x_2$, $x_3$, and $x_4$ of the polynomial $P(x)=x^4+ax^3+bx^2+cx+d$ are real.
  Find the smallest value the product $(x_1^2+1)(x_2^2+1)(x_3^2+1)(x_4^2+1)$ can take.
\end{example}

Here are two ways you could write the solution.\footnote{Former
solution worsened June 2018, with suggestions from Mitchell Lee.}

\subsection*{❌ Bad solution.}
$x_j^2+1 = (x-i)(x+i) \forall j$
$\implies \prod x_j^2+1 = \prod (x_j+i)(x_j-i) = P(i)P(-i)$
so $(b-d-1)^2 + (a-c)^2$.
$\because x_j = 1 \rightarrow 16$ and $\binom42-1 = 5$.
$b-d \ge 5$, so $\ge 16$.

\subsection*{✅ Good solution.}
The answer is $\boxed{16}$.
This can be achieved by taking $x_1 = x_2 = x_3 = x_4 = 1$,
whence the product is $2^4 = 16$, and $b-d = 5$.

We now show the quantity is always at least $16$.
We prove:
\begin{claim*}
  We always have $(x_1^2+1)(x_2^2+1)(x_3^2+1)(x_4^2+1) = (b-d-1)^2 + (a-c)^2$.
\end{claim*}
\begin{proof}
  Let $i = \sqrt{-1}$.
  The key observation is that
  \[ \prod_{j=1}^4 \left( x_j^2 + 1 \right)
    = \prod_{j=1}^4 (x_j - i)(x_j + i)
    = P(i)P(-i) = |P(i)|^2. \]
  Since $P(i) = (-1+b-d) + (c-a)i$, the claim follows.
\end{proof}
Since $b-d-1 \ge 4$, we get the desired lower bound of $4^2+0^2=16$.

\subsection*{💬 Commentary.}
These solutions have the same mathematical content.
But notice how in the better solution:
\begin{itemize}
  \ii The second solution makes it clear
  from the beginning what the answer is, and what the equality case is.
  (The first solution mixes these together.)
  \ii Moreover, the main idea (of factoring with $i$) is explicitly labeled,
  so that even if you have never seen the problem before,
  you can tell at a glance what the main idea of the solution is.
  \ii The equations are displayed in the second solution,
  making them much easier to read than in the first.
\end{itemize}
The second solution, despite being twice as ``long'',
is by far faster to read than the first solution.
In this case, the difference is not so bad because the
problem and solution are quite short.
However, in more involved problems the ``not-so-good solution''
becomes the ``completely unreadable solution''.


\subsection{Advanced example: USAMO 2014 Problem 4}
This example is meant to show what a more involved solution could look like.
(Beginners or even intermediate students shouldn't be discouraged
if it takes a while to understand even the ``good'' solution; it's not an easy problem.)

\begin{example}
  [USAMO 2014 Problem 4]
  Let $k$ be a positive integer.
  Two players $A$ and $B$ play a game on an infinite grid of regular hexagons.
  Initially all the grid cells are empty.
  Then the players alternately take turns with $A$ moving first.
  In her move, $A$ may choose two adjacent hexagons in the grid
  which are empty and place a counter in both of them.
  In his move, $B$ may choose any counter on the board and remove it.
  If at any time there are $k$ consecutive grid cells
  in a line all of which contain a counter, $A$ wins.
  Find the minimum value of $k$ for which $A$ cannot
  win in a finite number of moves, or prove that no such minimum value exists.
\end{example}

\subsection*{❌ Bad solution.}
Impose coordinates $(x,y)$.
If $B$ always removes counters from cells $(x,y)$ with $x + y \equiv 0 \pmod 3$,
then any line by $A$ gets interrupted so worst case $A$ gets up to $k=2+1+2=5$.
For $A$, make an equilateral triangle on second turn
and then put two more counters two away on third turn.
This makes a threat so $B$ removes an inner counter,
then $A$ puts this inner counter back it makes another threat each turn,
so $B$ has to keep doing this.
Eventually one of the inner counters is surrounded by a full hexagon,
so $A$ fills in the other gap,
which creates an double threat for $A$'s next turn so $A$ wins, answer $6$.

\subsection*{✅ Good solution.}
The answer is $k = 6$.

\paragraph{Proof that $A$ cannot win if $k=6$.}
We give a strategy for $B$ to prevent $A$'s victory.
Shade in every third cell, as shown in the figure below.
Then $A$ can never cover two shaded cells simultaneously on her turn.
Now suppose $B$ always removes a counter on a shaded cell
(and otherwise does whatever he wants).
Then he can prevent $A$ from ever getting six consecutive counters,
because any six consecutive cells contain two shaded cells.

\begin{center}
  \begin{asy}
    path hexagon = scale(0.5)*(dir(30)--dir(90)--dir(150)--dir(210)--dir(270)--dir(330)--cycle);
    unitsize(0.6cm);
    void mark(int x, int y) {
      pair P = dir(0)*x + dir(60)*y;
      filldraw(shift(P)*hexagon, purple, black+1);
    }
    void spot(int x, int y) {
      pair P = dir(0)*x + dir(60)*y;
      draw(shift(P)*hexagon);
    }
    int N = 4;
    for (int x=-N; x<=N; ++x) {
      for (int y=-N; y<=N; ++y) {
        if ( (abs(x+y)) > N ) continue;
        if (
          (x-y) % 3 == 0
        ) mark(x,y);
        else spot(x,y);
      }
    }
  \end{asy}
\end{center}

\paragraph{Example of a strategy for $A$ when $k=5$.}
We describe a winning strategy for $A$ explicitly.
Note that after $B$'s first turn there is one counter,
so then $A$ may create an equilateral triangle,
and hence after $B$'s second turn there are two consecutive counters.
Then, on her third turn,
$A$ places a pair of counters two spaces away on the same line.
Label the two inner cells $x$ and $y$ as shown below.
\begin{center}
  \begin{asy}
    path hexagon = scale(0.5)*(dir(30)--dir(90)--dir(150)--dir(210)--dir(270)--dir(330)--cycle);
    unitsize(0.6cm);
    void mark(int x, int y) {
      pair P = dir(0)*x + dir(60)*y;
      filldraw(shift(P)*hexagon, palecyan, black);
      filldraw(CR(P, 0.3), paleblue, blue);
    }
    void spot(int x, int y) {
      pair P = dir(0)*x + dir(60)*y;
      draw(shift(P)*hexagon);
    }
    for (int x=-3; x<=4; ++x) {
      for (int y=-2; y<=2; ++y) {
        if ( x+y < -3 ) continue;
        if ( x+y > 4 ) continue;
        spot(x,y);
      }
    }
    mark(-2, 0);
    mark(-1, 0);
    mark( 2, 0);
    mark( 3, 0);
    label("$x$", (-1)*dir(0));
    label("$y$", (2)*dir(0));
  \end{asy}
\end{center}
Now it is $B$'s turn to move;
in order to avoid losing immediately,
he must remove either $x$ or $y$.
Then on any subsequent turn, $A$ can replace $x$ or $y$
(whichever was removed)
and add one more adjacent counter.
This continues until either $x$ or $y$ has all its neighbors filled
(we ask $A$ to do so in such a way that
she avoids filling in the two central cells between $x$ and $y$
as long as possible).

So, let's say without loss of generality (by symmetry) that $x$
is completely surrounded by tokens.
Again, $B$ must choose to remove $x$ (or $A$ wins on her next turn).
After $x$ is removed by $B$, consider the following figure.
\begin{center}
  \begin{asy}
    path hexagon = scale(0.5)*(dir(30)--dir(90)--dir(150)--dir(210)--dir(270)--dir(330)--cycle);
    unitsize(0.6cm);
    void mark(int x, int y) {
      pair P = dir(0)*x + dir(60)*y;
      filldraw(shift(P)*hexagon, palecyan, black);
      filldraw(CR(P, 0.3), paleblue, blue);
    }
    void spot(int x, int y) {
      pair P = dir(0)*x + dir(60)*y;
      draw(shift(P)*hexagon);
    }
    for (int x=-4; x<=5; ++x) {
      for (int y=-3; y<=3; ++y) {
        if ( x+y < -4 ) continue;
        if ( x+y > 5 ) continue;
        spot(x,y);
      }
    }
    mark(-2, 0);
    mark( 0, 0);
    mark( 0, -1);
    mark(-1, 1);
    mark(-1, -1);
    mark(-2, 1);

    mark( 2, 0);
    mark( 2, 1);
    mark( 3, 0);
    label("$x$", (-1)*dir(0));
    label("$y$", (2)*dir(0));
    void win(int x, int y, pen p1, pen p2) {
      pair P = dir(0)*x + dir(60)*y;
      filldraw(shift(P)*hexagon, p1, p2);
    }
    win(1,0, lightgreen, deepgreen);
    win(0,1, lightgreen, deepgreen);
    win(0,2, lightred, red);
    win(0,3, lightred, red);
    win(4,0, lightred, red);
    win(5,0, lightred, red);
  \end{asy}
\end{center}
We let $A$ play in the two marked green cells.
Then, regardless of what move $B$ plays,
one of the two choices of moves marked in red lets $A$ win.
Thus, we have described a winning strategy when $k=5$ for $A$.

\subsection*{💬 Commentary.}
The bad solution doesn't give definitions of any of the objects it is using,
such as how the coordinates $(x,y)$ are specified,
or the meanings of any of the made-up terms like ``line by $A$'',
``threat'', ``inner counter'', ``worst case'', ``surrounded by a full hexagon'',
``fills in the other gap'', ``double threat''.
The reader is forced to guess.

Diagrams are absent, and there are no names besides the $k$ given in the problem
(which is abused slightly as $k = 2+1+2 = 5$).
It also never clearly states what the answer is,
and smashes everything into one paragraph.

You can read lots of solutions to this problem on AoPS at
\url{https://aops.com/community/p3478584}
if you want to see more examples of good and bad writing.

\newpage

\section{Concrete advice}
Here are some tips of mine that I don't think are stressed enough.

\subsection{Never write wrong math}
This is much more of a math issue than a style issue:
you can lose all of your points for making false claims.
Personally, I often \emph{stop reading} a solution if it makes
an egregiously false claim: if someone claims that some fixed
point is the incenter of $ABC$,
when it's actually the arc midpoint,
then I know the solution isn't going to have any substantial progress.

As a special case, don't say something that is partially true
and then say how to fix it later.
At best this will annoy the grader;
at worst they may get confused and think the solution is wrong.

\subsection{Emphasize the point where you cross the ocean}
Solutions to olympiad problems often involve a few key ideas,
with the rest of the solution being checking details.
You want graders to immediately see all the key ideas in the solution:
this way, they quickly have a high-level understanding of your approach.

Let me share a quote from Scott Aaronson:
\begin{quote}
  Suppose your friend in Boston blindfolded you,
  drove you around for twenty minutes,
  then took the blindfold off and claimed you were now in Beijing.
  Yes, you do see Chinese signs and pagoda roofs,
  and no, you can't immediately disprove him ---
  but based on your knowledge of both cars and geography,
  isn't it more likely you're just in Chinatown?
  \dots
  \textbf{We start in Boston, we end up in Beijing,
  and at no point is anything resembling an ocean ever crossed.}
\end{quote}
Olympiad solutions work the same way:
a geometry solution might require a student to do some angle chasing,
use \href{https://w.wiki/AEpo}{the incenter-excenter lemma} to deduce that two triangles are congruent,
and then finish by doing a little more angle chasing.
In that case, you want to highlight the key step of proving the two triangles
were congruent, so the grader sees it immediately and can say
``okay, this student is using this approach''.

Ways that you can highlight this are:
\begin{itemize}
  \ii \alert{Isolating crucial steps and claims}
  as their own self-contained parts.\footnote{This is often useful for another reason:
    breaking the proof into individual steps.
    The complexity of understanding a proof grows super-linearly
    in its length; therefore breaking it into smaller chunks
    is often a good thing.}
  \ii Using \alert{claims} to say what you're doing.
  Rather than doing angle chasing and writing
  ``blah blah blah, therefore $\triangle M_B I_B M \sim \triangle M_C I_C M$'',
  consider instead ``We claim $\triangle M_B I_B M \sim \triangle M_C I_C M$, proof''.
  \ii \alert{Displaying important equations}.
  For example, notice how the line
  \begin{equation}
    \triangle M_B I_B M \sim \triangle M_C I_C M
    \label{eq:displaytri}
  \end{equation}
  jumps out at the reader.
  You can even number such claims to reference them later,
  e.g.\ ``by \eqref{eq:displaytri}''.
  This is especially useful in functional equations.
  \ii \alert{Just say it}!
  Little hints like ``the crucial claim is $X$''
  or ``the main idea is $Y$'' are immensely helpful.
  Don't make $X$ and $Y$ look like another intermediate step.
\end{itemize}

In the good USAMO 2014/1 sample, we can see two examples of this.
First, the critical identity
\[ (x_1^2+1)(x_2^2+1)(x_3^2+1)(x_4^2+1) = (b-d-1)^2 + (a-c)^2 \]
was written out as a separate independent claim, with a self-contained proof.
This divorces the ``prove an identity'' part of the problem from the
``prove the lower bound'' part of the problem,
splitting the solution into two independent halves.
The reader doesn't need to consider $b-d \geq 5$ in the first half;
this compartmentalization helps make things easier.
Second, the introduction of $i$ on its separate display,
explicitly telegraphed with ``the key observation'',
makes it obvious to the reader that this particular step is the heart of the proof.

\subsection{``Find all\dots'' and ``find the minimum/maximum\dots''}
Many problems will ask you to ``find all objects satisfying some condition''
(for example, functional equations, Diophantine equations).
For any solution of this form, I strongly recommend that you
structure your solution as follows:
\begin{itemize}
  \ii Start by writing ``\alert{We claim the answer is \dots}''.
  \ii Then, say ``\alert{We prove these satisfy the conditions}'', and do so.
  For example, in a functional equation with answer $f(x) = x^2$,
  you should plug this $f$ back in and verify the equation is satisfied.
  Even if this verification is trivial, you must still explicitly include it,
  because it is part of the problem.
  \ii Finally, say ``\alert{Now we prove these are the only ones}'' and do so.
\end{itemize}

Similarly, some problems will ask you to ``find the minimum/maximum value of $X$''.
In such situations, I strongly recommend you write your solution as follows:
\begin{itemize}
  \ii Start by writing ``\alert{We claim the minimum/maximum is \dots}''.
  \ii Then, say ``\alert{We prove that this is attainable}'',
  and give the construction (or otherwise prove existence).
  Even if this verification is trivial, you must still explicitly include it,
  because it is part of the problem.
  \ii Finally, say ``\alert{We prove this is a lower/upper bound}'', and do so.
\end{itemize}

Failing to do one of the steps mentioned above is a classic newbie mistake.
Make it abundantly clear to the grader that you know the difference
between a bound and a maximum.
The good solutions to both problems on USAMO 2014 follow this structure.

\subsection{Leave space}
Most people don't leave enough space. This makes solutions hard to read.

Examples of things you can do:
\begin{itemize}
  \ii \alert{Skip a line after paragraphs}.
  Use paragraph breaks more often than you already do.

  \ii If you isolate a specific \alert{lemma or claim} in your proof,
  then it should be on its own line,
  with some whitespace before and after it.

  \ii Any time you do casework,
  you should always \alert{split cases into separate paragraphs or bullet points}.
  Make it visually clear when each case begins and ends.

  \ii Display important equations, rather than squeezing them into paragraphs.
  If you have a long calculation,
  then do an aligned display\footnote{This is the
    \texttt{align*} environment, for those of you that like \LaTeX.}
  rather than squeezing it into a paragraph.
  For example, instead of writing $0 \le {(a - b)^2} = {(a + b)^2} - 4ab = {(10 - c)^2} - 4\left( {25 - c(a + b)} \right) = {(10 - c)^2} - 4\left( {25 - c(10 - c)} \right) = c(20 - 3c)$, write instead
  \begin{align*}
    0 &\le {(a - b)^2}\\
     &= {(a + b)^2} - 4ab\\
     &= {(10 - c)^2} - 4\left( {25 - c(a + b)} \right)\\
     &= {(10 - c)^2} - 4\left( {25 - c(10 - c)} \right)\\
     &= c(20 - 3c).
  \end{align*}
\end{itemize}

\subsection{For the love of god please include a diagram if you needed one}
If you used a diagram to solve the problem, give one to the reader too.
(This is not limited to just geometry problems.)

In competition settings, you do \emph{not} need to draw a new diagram for the solution.
Just turn in the one that you already drew when solving the problem,
by adding the appropriate page headers to it.

``Why can't the reader draw their own diagram in Geogebra?''
Well, yes, in theory. But theory is not the same as practice.
For example, suppose you write ``let $M$ be the midpoint of $\ol{BC}$''
when you meant ``let $M$ be the midpoint of $\ol{EF}$''.
Now because of that one typo, you're \emph{completely screwed},
because you \emph{also} forced the reader to draw their own diagram,
which is now different from yours.
The reader has no way to fix your mistake for you.
There's no reason to press your luck like this; just include a diagram.

If you want to understand how painful it is to read a solution with no diagram provided,
check out the AoPS thread for USAMO 2014/4 at \url{https://aops.com/community/p3478584}
and try reading some of the text-only solutions.

\subsection{Pay special attention to definitions (extra important)}
The previous illustration of how the typo
``let $M$ be the midpoint of $\ol{BC}$'' can kill you
illustrates a more general point: you have to get definitions right.
For intermediate logic, a reader can rely on truth/falsity of statements
to help fill in gaps or fix typos --- but definitions cannot be
intrinsically right or wrong, so the reader has to use the one you gave them.

Thus it's absolutely critical your definitions are solid.
Whenever you define a new object or algorithm,
pay special care that your specifications are complete and correct.

Tips:
\begin{itemize}
  \ii \alert{Write in complete sentences}.
  Do not skimp on words when you're defining objects, they're too important.
  (If you want to be lazy, do so in the main body of your solution, not the definitions.)

  \ii If the definition is complicated enough, \alert{give some examples}.
  (To be honest, I always thought the examples were worth more than the formal definition.)
  Compare for example
  \begin{quote}
  \slshape
  Define a \textbf{block} of letters to be a maximal
  contiguous subsequence of consecutive letters.
  \end{quote}
  versus
  \begin{quote}
  \slshape
  Define a \textbf{block} of letters to be a maximal
  contiguous subsequence of consecutive letters.
  For example, the word $aabbbcaaa$ has four blocks,
  namely $aa$, $bbb$, $c$, $aaa$, appearing in that order.
  \end{quote}

  \ii If you are defining an object,
  \alert{consider whether it makes sense to name it}.
  For example, if you find yourself writing ``the set of bad numbers'' over and over,
  consider naming it $B$ instead.\footnote{One subtle
    side effect of this habit is that it will also remind you to check
    that you \emph{gave} a definition of ``bad number'' to begin with.
    If you try to name an object you never gave a full definition of,
    it will tend to jump out at you.}
  As another example, consider
  \begin{quote}
  \slshape
  Given a word $A$, we introduce the following notation for its $m$ blocks:
  \[ A =
    A_1 A_2 \dots A_m
    = \underbrace{a_1 \dots a_1}_{x_1}
    \underbrace{a_2 \dots a_2}_{x_2}
    \dots
    \underbrace{a_m \dots a_m}_{x_m}.
  \]
  \end{quote}

  \ii \alert{Never write the words ``worst case''} unless you really,
  really know what you are doing.
  This term is tantamount to saying to the reader,
  ``I'm only going to consider this case, but I won't explain why'',
  and is quite commonly accompanied by false or circular reasoning.
\end{itemize}
The examples for USAMO 2014/4 shows a night-and-day contrast.
In the bad write-up, nothing is defined, illustrated, or named.
In the good write-up, all named objects are explicitly shown.

\subsection{Other things}
Try to have nice handwriting.
Leave 1-inch (or more) margins.
% Try to avoid making typos.
Write your proofs forwards even if you solved the problem backwards.
If you need to cite a theorem, say clearly how you're doing so.
Use variable names at your discretion.
Strike out and cross out unwanted parts of your solution (don't scribble).

I'm sure someone has told you these before.
If not, consider reading
\url{https://www.aops.com/articles/how-to-write-solution}.

\section{Funny quote from \href{https://www.gleech.org/better-maths}{Steven Witten}}
\begin{quote}
  \slshape
  Imagine I asked you to learn a programming language where:
  \begin{itemize}
  \ii All the variable names were a single letter,
  and where programmers enjoyed using foreign alphabets,
  glyph variation and fonts to disambiguate their code from meaningless gibberish.
  \ii None of the functions were documented,
  and instead the API docs consisted of circular references to other pieces of similar code,
  often with the same names overloaded into multiple meanings, often impossible to Google.
  \ii None of the sample code could be run on a typical computer;
  in fact, most of it was pseudo-code lacking a definition of input and output,
  or even the environment it was supposed to run.
  \end{itemize}
\end{quote}

\appendix

\section{Notes specific to mathematical competitions}
Up until now I've given my advice for how to write solutions well.
But I know a lot of you are specifically interested in olympiad grading,
so here are a few quick remarks to that end.
These comments are meant for USA(J)MO in particular
but should apply to other respectable contests as well.

\subsection{Grading}
Your score on an olympiad problem is a nonnegative integer at most $7$.
The unspoken rubric reads something like the following:
\begin{center}
\begin{tabular}[h]{ll}
  & Description \\ \hline
  \boldmath{$7^\ast$} & Problem was solved \\
  $6$ & Tiny slip (and contestant could repair) \\
  $5$ & Small gap or mistake, but non-central \\ \hline
  $2$ & Lots of genuine progress \\
  \boldmath{$1^\ast$} & Significant non-trivial progress \\
  \boldmath{$0^\ast$} & ``Busy work'', special cases, lots of writing
\end{tabular}
\end{center}
The ``default'' scores are starred above.
Note that, unlike high school English class or the SAT essay,
you don't get points just because you wrote a lot!

In theory, your solutions to olympiads are graded solely based on math.
In practice, style still does play a role in some ways:
the harder your solution is to understand,
the less likely the grader is to understand you,
and the less likely you are to earn points you deserve.\footnote{In addition,
  poorly written solutions make the graders sad, and you wouldn't want that, would you?}

\subsection{More examples of decent write-ups}
I should note that on my website
\begin{center}
  \url{https://web.evanchen.cc/problems.html}
\end{center}
there are a large number of solutions written by me
to past problems on the USAMO, IMO, USA TST(ST), etc.
In particular, all USAMO and IMO problems since the year 2000
are present.

Not all the solutions are complete (some of them are just outlines),
but I think the majority of them are full write-ups,
and these can help give more examples of solutions
that you can compare to or model your own work after.

\subsection{How much detail to include}
A common question I get is what the minimum amount of detail needed
to get full marks for a solution is.
The answer is simple: enough to convince the grader you solved the problem.

There is a myth that, sort of like your high school English or math teacher,
you can lose points for ``not writing enough'' or not having certain key words
or leaving out details that were obvious to everyone.
This is not really how it works.
USAMO graders are interested in whether you solved the problem
rather than your ability to fill pages with ink.

Basically, \alert{you lose points if a student who did NOT solve the problem
could have written the same words as you}.
For example, whenever you say something like ``it's easy to see $X$'',
the grader has to ask whether you actually understand why $X$ is true,
or don't know and are just bluffing.
So that's always the criteria you should have in your head
when deciding to write out in full.

\subsection{Citing lemmas}
In general it is usually okay to cite a result that is
(i) named, and (ii) does not trivialize the given problem.
Anything outside this scope is a ``gray area''
and I don't want to commit to a hard set of guidelines.

However, the main thing I want to say is that
\alert{if in doubt, outline a proof}.
You don't have to choose between the extremes
``say absolute zero'' and ``prove quoted lemma in full gory detail''.
It's better to just include a couple lines giving the overall idea of the proof
to show that you \emph{could} write it out if you wanted to,
but are omitting it because the result is already known.

\subsection{Fake-solving problems}
With all that said, I would say in the end,
when \alert{people don't get the points they expect,
it's because their solution is actually wrong or incomplete},
not because they wrote it poorly.
This is true something like 90\% of the time, maybe more.

Some common ways to lose most or all of your points
by virtue of not having solved the problem:
\begin{itemize}
  \ii Flipping an inequality sign.
  \ii Not understanding what the word ``function'' means
  in a functional equation.
  \ii Making some assumption that seems intuitive,
  but actually requires justification (and is the main difficulty of the problem).
  \ii Stating key assertions with no proof
  (often which are equivalent to the problem).
  \ii Making some actual logical error
  (for example, the so-called ``pointwise trap'').
  \ii Missing some case or possibility that the student didn't realize existed.
  \ii Not understanding the problem statement altogether
  (for example, not knowing that ``find all'' problems have two parts,
  and only doing one direction).
\end{itemize}
Some examples of USAMO problems that are notorious for generating wrong solutions:
USAMO 2003/6, USAMO 2007/2, USAMO 2010/3, USAMO 2016/4.

I should say there is no shame in having a wrong solution to a problem,
it really happens to everyone more often than anyone wants to admit.
Just don't delude yourself into thinking that you lost points
you deserved because the graders didn't like your style.

\section{Checking your work}
\subsection{How to check your solutions during the year}
When you are practicing during the year,
the best way to get feedback on proofs is to have a friend/coach
who can check your work and offer suggestions.
But the supply of people willing to do this is admittedly low,
so most people are not so lucky to have access to feedback.
Almost everyone gets by instead with the following algorithm:
\begin{description}
  \ii[Step 1.] Write up your solution neatly.
  The more clean your write-up is, the more likely you are to catch
  your own mistakes\footnote{If you find that most of your solutions
    still have mistakes which aren't caught at this first step,
    this suggests you are either not writing well
    or the problems you are trying are too hard for you.}.
  As you write your solution, actively look for ways to
  reorganize, consolidate, or simplify your solution
  (see \href{https://web.evanchen.cc/faq-contest.html#C-24}{Evan's FAQ C-24}).

  \ii[Step 2.] Once you have a draft of a solution written,
  find the problem's thread on the Art of Problem Solving forum.
  You can find almost every past contest problem via the Contest Index at
  \url{https://aops.com/community/c13}.\footnote{I
    do NOT recommend using the AoPS Wiki in place of the Contest Index.
    The solution quality in the wiki is generally much poorer than the forum.}

  Compare your solution draft to others posted,
  particularly reputable\footnote{If you don't know who is ``reputable'' yet,
    using the cleanliness of the presentation is an astonishingly good proxy.
    People who typeset their LaTeX correctly,
    isolate claims/proofs and sections, actually include diagrams, etc.
    are (not-so-coincidentally) usually the ones who know what they're doing.
    As a concrete example, if you were looking for a solution to USAMO 2014/4,
    \href{https://aops.com/community/p15752333}{post \#68 by TheUltimate123}
    is the kind of post you should read first: properly drawn diagram,
    clear headings showing the two separate parts of the proof,
    proper formatting, and so on.} users.
  Often, a problem will have essentially only a few approaches,
  and you'll find another user who had more or less
  the same approach.\footnote{There are unfortunately some problems,
    like USAMO 2017/1, where so many different solutions are possible
    that any two people are likely to have different approaches.}
  This serves as a sanity check that what you have does work.

  If you find your solution is way shorter or simpler
  than everyone else, then you have good reason to be suspicious.
  Look for the ocean-crossing point in other people's solutions.
  Why did they have to work so hard there, while you did not?
  Often, that's where the mistake will be.

  \ii[Step 3.] After any polishes or edits inspired by other's works,
  post your final draft on the forum thread too.
  By Cunningham's Law, if you have a blatantly wrong solution,
  someone will often point it out within a few hours.
\end{description}

\subsection{How to check your solutions during a contest}
Of course, it is critical to eventually
be able to check your own work independently
without consulting other people.
The IMO does not have live feedback;
by the time someone tells you about a mistake, it is too late!

If you are a beginner it might take a while to reach this stage,
but you should set this as a goal for where you want to end up.
It is easier than you might expect ---
as you naturally get better at solving problems, your instincts
about the correctness of proofs will automatically develop too.

During the contest, the only advice I have is
``write clearly and carefully''
(which is why developing these habits pays off later).
I cannot tell you how many times I realized only during the write-up phase
that the ``solution'' I thought I had was actually flawed.
\end{document}
