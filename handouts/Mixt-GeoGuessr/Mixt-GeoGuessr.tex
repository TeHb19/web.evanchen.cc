\documentclass[11pt]{scrartcl}
\usepackage[sexy]{evan}

\begin{document}
\title{A Guessing Game: Mixtilinear Incircles}
\date{11 August 2015}
\maketitle

\begin{center}
  \itshape
  Sometimes figuring out what to prove is harder than actually proving it!
\end{center}

An important skill for olympiad geometers is to be able to guess when three points are collinear,
four points are concyclic, three lines are concurrent, and so on.
Difficult geometry problems (on the level of IMO 3/6) often amount to finding two or three critical claims;
each of these claims may be no harder to prove than an IMO 1/4, but making the right guesses of \emph{what} to prove
can turn out to the core difficulty of the problem.
For a fantastic example, see my solution to
\href{http://www.aops.com/community/c6h418983p3518149}{IMO 2011/6}.

In this exercise, I'll put write down a configuration of several points, lines, and circles.
Your job is to find as many ``coincidences'' as you can: nontrivial collinearities, equal angles,
cyclic quadrilaterals, and so on and so forth.
Then, see if you can prove them!

\tableofcontents

\section{The Configuration}
\begin{mdframed}
  Let $ABC$ be an acute triangle with incenter $I$ and circumcircle $\Gamma$,
  and let $D$ and $E$ be the contact points of the incircle and $A$-excircle on $BC$.
  Let $M_A$, $M_B$, $M_C$ be the midpoints of the arcs $BC$, $CA$, $AB$ of $\Gamma$.

  The \textbf{$\boldmath A$-mixtilinear incircle} is the circle $\omega_A$ which is
  tangent to $AB$, $AC$, $\Gamma$ at points $B_1$, $C_1$, $T$.
\end{mdframed}

On the next page there is a bare-bones diagram with all these points,
as well as some hints to get you started.
However, before using it, \textbf{I encourage you to try and find as many things as you can
using your own ruler and compass.}
On an olympiad, you do not have the luxury of referring to a perfect, computer-drawn diagram!
You can (and should) use more than one hand-drawn diagram.

\eject

\section{Some Hints}
Now that you've taken a look with the hand-drawn diagram,
see if you can spot even more things in the following accurate computer-generated figure.

\begin{center}
\begin{asy}
unitsize(5cm);

pair A = dir(140);
pair B = dir(210);
pair C = dir(330);
pair M_A = dir(270);
pair M_B = dir(55);
pair M_C = dir(175);

pair I = incenter(A, B, C);
pair D = foot(I, B, C);
pair E = B+C-D;
pair B_1 = extension(I, I+dir(90)*dir(A-I), A, B);
pair C_1 = extension(I, I+dir(90)*dir(A-I), A, C);
pair T = extension(M_C, B_1, M_B, C_1);

draw(unitcircle);
draw(A--M_A);

draw(circumcircle(T, B_1, C_1), blue);
draw(A--B--C--cycle);

dot("$A$", A, dir(A));
dot("$B$", B, dir(B));
dot("$C$", C, dir(C));
dot("$M_A$", M_A, dir(M_A));
dot("$M_B$", M_B, dir(M_B));
dot("$M_C$", M_C, dir(M_C));
dot("$I$", I, dir(I));
dot("$D$", D, dir(D));
dot("$E$", E, dir(E));
dot("$B_1$", B_1, dir(B_1));
dot("$C_1$", C_1, dir(C_1));
dot("$T$", T, dir(T));

/* TSQ Source:

!unitsize(5cm);

A = dir 140
B = dir 210
C = dir 330
M_A = dir 270
M_B = dir 55
M_C = dir 175

I = incenter A B C
D = foot I B C
E = B+C-D
B_1 = extension I I+dir(90)*dir(A-I) A B
C_1 = extension I I+dir(90)*dir(A-I) A C
T = extension M_C B_1 M_B C_1

unitcircle
A--M_A

circumcircle T B_1 C_1 blue
A--B--C--cycle

*/
\end{asy}
\end{center}

Here are some possible hints for things you could look for:
\begin{itemize}
  \ii There are at least three nontrivial collinearities among the labelled points.
  \ii There are at least two nontrivial cyclic quadrilaterals among the labelled points.
  \ii There are several nontrivial pairs of equal angles among the labelled points.
  \ii There is at least one set of concurrent lines (which meet outside the triangle).
  \ii Look at the ``top'' of the circumcircle.
  \ii There are at least two lines tangent to some circumcircles.
\end{itemize}

My list of properties has ten items.
When you want to see my answers, turn the page.

\eject

\section{Answers}
\begin{center}
\begin{asy}
unitsize(5cm);

pair A = dir(140);
pair B = dir(210);
pair C = dir(330);
pair M_A = dir(270);
pair M_B = dir(55);
pair M_C = dir(175);

pair I = incenter(A, B, C);
pair D = foot(I, B, C);
pair E = B+C-D;

pair B_1 = extension(I, I+dir(90)*dir(A-I), A, B);
pair C_1 = extension(I, I+dir(90)*dir(A-I), A, C);

pair T = extension(M_C, B_1, M_B, C_1);
draw(C--M_C, heavygreen);
draw(B--M_B, heavygreen);

filldraw(unitcircle, opacity(0.02)+cyan, black);

draw(A--M_A);

filldraw(circumcircle(T, B_1, C_1), opacity(0.05)+lightblue, blue);
markangle(B,A,T,heavycyan);
markangle(E,A,C,heavycyan);
markangle(A,T,B,heavymagenta);
markangle(C,T,D,heavymagenta);

draw(A--B--C--cycle);

draw(T--M_C, red+dashed);
draw(T--M_B, red+dashed);

draw(A--E, blue);
pair X = dir(90);
draw(T--X, blue);

draw(B--T--C, magenta);
draw(A--T--D, magenta);

filldraw(circumcircle(T, D, M_A), opacity(0.1)+lightred, orange);

pair Z = extension(B, C, T, M_A);
draw(Z--C_1, lightgreen);
draw(Z--B, lightgreen);
draw(Z--M_A, lightgreen);

filldraw(circumcircle(B, B_1, T), opacity(0.04)+green, heavygreen);
filldraw(circumcircle(C, C_1, T), opacity(0.04)+green, heavygreen);

pair H = extension(A, D, T, M_A);
draw(A--H, grey+dashed);

dot(H);
dot(extension(A,E,X,I));
dot(extension(T,M_A,B,C));
dot(extension(A,M_A,B,C));
dot(X);

dot("$A$", A, dir(A));
dot("$B$", B, dir(B));
dot("$C$", C, dir(C));
dot("$M_A$", M_A, dir(M_A));
dot("$M_B$", M_B, dir(M_B));
dot("$M_C$", M_C, dir(M_C));
dot("$I$", I, dir(I));
dot("$D$", D, dir(D));
dot("$E$", E, dir(E));
dot("$B_1$", B_1, dir(B_1));
dot("$C_1$", C_1, dir(C_1));
dot("$T$", T, dir(T));

/* TSQ Source:

!unitsize(5cm);

A = dir 140
B = dir 210
C = dir 330
M_A = dir 270
M_B = dir 55
M_C = dir 175

I = incenter A B C
D = foot I B C
E = B+C-D

B_1 = extension I I+dir(90)*dir(A-I) A B
C_1 = extension I I+dir(90)*dir(A-I) A C

T = extension M_C B_1 M_B C_1
C--M_C heavygreen
B--M_B heavygreen

unitcircle 0.02 cyan / black

A--M_A

circumcircle T B_1 C_1 0.05 lightblue / blue

A--B--C--cycle

T--M_C red dashed
T--M_B red dashed

A--E blue
X := dir 90
T--X blue

B--T--C magenta
A--T--D magenta

circumcircle T D M_A 0.1 lightred / orange

Z := extension B C T M_A
Z--C_1 lightgreen
Z--B lightgreen
Z--M_A lightgreen

circumcircle B B_1 T 0.04 green / heavygreen
circumcircle C C_1 T 0.04 green / heavygreen

H := extension A D T M_A
A--H grey dashed

!dot(H);
!dot(extension(A,E,X,I));
!dot(extension(T,M_A,B,C));
!dot(extension(A,M_A,B,C));
!dot(X);

*/
\end{asy}
\end{center}

\begin{enumerate}
  \ii Points $T$, $B_1$, $M_C$ are collinear.
  Similarly, points $T$, $C_1$, $M_B$ are collinear.
  \ii Point $I$ is the midpoint of $B_1C_1$.
  \ii Ray $TI$ passes through the ``topmost'' points of both $\omega_A$ and $\Gamma$
  (the point opposite $M_A$ on $\Gamma$).
  In particular, $AE$ and $TI$ meet on $\omega_A$.
  \ii $\angle BAT = \angle CAE$, and equivalent angles.
  \ii $\angle ATB = \angle CTD$, and equivalent angles.
  \ii $\angle ATM_C = \angle M_BTI$ (not shown in figure).
  \ii Quadrilaterals $BB_1IT$ and $CC_1IT$ are cyclic.
  \ii Lines $BI$ and $CI$ are tangents to the previously mentioned quadrilaterals.
  \ii The intersection of lines $AI$ and $BC$ lies on the circumcircle of $\triangle DTM_A$.
  \ii Lines $BC$, $TM_A$, $B_1C_1$ are concurrent.
  \ii Lines $TM_A$ and $AD$ meet on the mixtilinear incircle. (Thanks to my student H.W.\ for finding this one!)
\end{enumerate}
This list is by no means exhaustive --- there are more properties buried in here that I haven't mentioned.

\section{Sketches of Solutions}
\begin{enumerate}
  \ii Consider the homothety at $T$ sending $\omega_A$ to $\Gamma$.
  \ii You can use Pascal's Theorem on $\Gamma$.
  (The special case when $AB=AC$ was IMO 1978).
  \ii In $\triangle TB_1C_1$, $TA$ is a symmedian and $TI$ is a median.
  (This was given as a problem on Iran 2002.)
  \ii Inversion at $A$. (This was EGMO 2013, Problem 2.)
  \ii Reflect $T$ across the perpendicular bisector of $BC$.
  \ii This just follows by symmedians.
  \ii Angle chasing (using above properties).
  \ii Angle chasing (using above properties).
  \ii Some more angle chasing, using $\angle ATB = \angle CTD$.
  \ii Use Pascal on $\Gamma$ once again.
  \ii Two homotheties.
\end{enumerate}
For more detailed discussion, you might consult \url{http://blog.evanchen.cc/2015/08/11/the-mixtilinear-incircle}.

\vspace{2em}
Want to try this again?
See what coincidences you can find if you draw in all three mixtilinear incircles.
For example, what are their radical axii?
Can you find some concurrent cevians?


\end{document}
